\subsection*{Números Complexos}


\begin{frame}{Uma breve introdução aos números complexos\footnotemark}
\footnotetext[1]{Veja o apêndice do livro James Stewart, Cálculo Volume 1.}

 O conjunto dos \dt{números complexos}, denotado por  $\mathbb{C}$, são formados pelos elementos da forma
\[z=a+bi,\ a,b\in \R, \text{ com }  i^2=-1,\]
onde estão definidas as operações de adição e multiplicação:
\begin{itemize}
\item $(a+bi)+(c+di)=(a+c)+(b+d)i$
\item $(a+bi)\cdot(c+di)=(ac-bd)+(ad+bc)i$
\end{itemize}

\begin{exe}
 Calcule $(-1+3i)(2-5i)$
\end{exe}

\end{frame}


\begin{frame}
O  \dt{módulo} de $z$ é definido por 
\[|z|:=\sqrt{a^2+b^2}.\] 

O \dt{Complexo Conjugado} de $z=a+bi$ é 
\[\bar{z}=a-bi.\]


\begin{block}{Propriedades}
\begin{enumerate}
\item $z\bar{z}=|z|^2$.
\item $\overline{z+w}=\bar{z}+\bar{w}$,\ \ \ \ $\overline{zw}=\bar{z}\bar{w}$\ \ \ \ e\ \ \ \ $\overline{z^n}=\bar{z}^n$

\item Se $z$ é raíz de um polinômio, então $\bar{z}$ também o é.
\end{enumerate}
\end{block}



\end{frame}

\begin{frame}
\begin{exe}
\begin{enumerate}
\item Expresse o número $\frac{-1+3i}{2+5i}$ na forma $a+bi$.

\item Encontre as raízes da equação $x^2+x+1$
\end{enumerate}
\end{exe}

\begin{teo}[Teorema Fundamental da Álgebra]
Toda equação polinômial 
\[a_nx^n+a_{n-1}x^{n-1}+\cdots a_1x+a_0=0,\]
onde $a_k\in \R$ e $n\geq 1$, tem exatamente $n$ soluções no conjunto dos números complexos.
\end{teo}
\end{frame}

\begin{frame}{Forma Polar}

Qualquer número complexo $z$ pode ser escrito na {\color{blue} forma polar}
\[z=r(\cos(\theta)+i\sen(\theta)),\]
onde $r=|z|$ e $\tan(\theta)=b/a$. O ângulo $\theta$ é chamado \dt{argumento} de $z$.

\begin{exe}
Escreva $z=1+i$ e $w=i$ na forma polar.
\end{exe}


\end{frame}

\begin{frame}{Propriedades}
Sejam ${\color{blue}z_1}={\color{blue}r_1}(\cos({\color{blue}\theta_1})
+i\sen({\color{blue}\theta_1}))$ ${\color{red}z_2}={\color{red}r_2}(\cos({\color{red}\theta_2})
+i\sen({\color{red}\theta_2}))$. Então
\[{\color{blue}z_1}{\color{red}z_2}={\color{blue}r_1}{\color{red}r_2}
\left(\cos({\color{blue}\theta_1}+{\color{red}\theta_2})
+i\sen({\color{blue}\theta_1}+{\color{red}\theta_2})\right),\]
\[\frac{{\color{blue}z_1}}{{\color{red}z_2}}
=\frac{{\color{blue}r_1}}{{\color{red}r_2}}
\left(\cos({\color{blue}\theta_1}-{\color{red}\theta_2})
+i\sen({\color{blue}\theta_1}-{\color{red}\theta_2})\right),\]
\begin{block}{Fórmula de De Moivre}
Se $z=r(\cos(\theta)+i\sen(\theta))$, então
\[z^n=r^n(\cos(n\theta)+i\sen(n\theta)), \ \forall n\in \mathbb{N}.\]
\end{block}
\end{frame}


\begin{frame}
\begin{exe}
\begin{enumerate}
\item Ache o produto dos números $z=1+i$ e $w=\sqrt{3}-i$ na forma polar.

\item Ache $\left(\frac{1}{2}+\frac{1}{2}i\right)^{10}$.

\item O que significa geometricamente multiplicar um número complexo $z$ por $i$?

\item Calcule $\sqrt[3]{1}$ em  $\mathbb{C}$, isto é, os valores de $z\in \mathbb{C}$ tais que $z^3=1$.
\end{enumerate}
\end{exe}
\end{frame}