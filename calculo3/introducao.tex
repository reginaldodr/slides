\section{Introdução}


\begin{frame}[label=introducao]
\frametitle{Introdução}

\begin{block}{Otimização de uma variável}
Imagine que você foi contratado por uma empresa que fabrica caixas sem tampa. Cada caixa é construída a partir de uma folha retangular de papelão medindo 30cm x 50cm. Para se construir uma caixa, um quadrado de lado medindo $x$ cm é retirado de cada canto da folha de papelão. O problema é determinar o valor de $x$ a fim de que a caixa correspondente tenha \dt{maior volume possível}. 
\end{block}


%\begin{center}
%\psset{unit=8mm}
%\begin{pspicture}(0,0)(7,4.5)
%\psline(1,1)(6,1)(6,4)(1,4)(1,1)
%\psline[linestyle=dashed](2,1)(2,2)(1,2)
%\psline[linestyle=dashed](5,1)(5,2)(6,2)
%\psline[linestyle=dashed](1,3)(2,3)(2,4)
%\psline[linestyle=dashed](6,3)(5,3)(5,4)
%\psline{|-|}(1,0.7)(6,0.7)
%\psline{|-|}(0.7,1)(0.7,4)
%\put(-0.5,2.5){30cm}
%\put(3,0.3){50cm}
%\psline{|-|}(6.3,3)(6.3,4)
%\psline{|-|}(5,4.3)(6,4.3)
%\put(6.4,3.5){$x$}
%\put(5.5,4.4){$x$}
%\end{pspicture}
%\end{center}

\begin{alertblock}{Problema de Maximização}
\dt{Função-objetivo:} 
\[\dt{V(x)=(30-2x)(50-2x)x}\] 
sujeito à restrição: $\dt{0\leq x\leq 15}$.
\end{alertblock}

\end{frame}



\begin{frame}

\begin{block}{Otimização de várias variáveis}
Agora imagine que você foi contratado por  uma empresa que fornece refeições a seus funcionários.
\smallskip

A fim de estabelecer um cardápio que atenda {\color{blue}a quantidade diária mínima de vitaminas} que um funcionário deve consumir  e {\color{red} minimize o custo de compras dos  diversos tipos de alimentos}.
\smallskip


Um nutricionista forneceu uma tabela que especifica a {\color{blue} quantidade mínima de cada tipo de vitamina que deve ser ingerida diariamente} por cada funcionário:
\end{block}

\begin{center}

\begin{tabu}{|l|c|c|c|c|}
\hline
Tipo de vitamina & A & B & C & E\\ \hline
	\rowfont{\color{blue}} Quantidade  mínima (mg) & 3 & 1.1 & 60 & 11\\ \hline
\end{tabu} 
\end{center}


\end{frame}

\begin{frame}
Ele também forneceu uma tabela com a quantidade (em mg) de vitaminas em cada 100 gramas de 9 tipos de alimentos diferentes,
\begin{center}
\begin{tabu}{|c|c|c|c|c|}
\hline
 Alimento & A & B & C & E  \\ \hline
1 & 0.140 & 0.08 & 0 & 1.60 \\ \hline
2 & 0.580 & 0 & 0 & 0  \\ \hline
3 & 0.150 & 1.30 & 26 & 6.90 \\ \hline
4 & 0 & 0.08 & 38 & 0.2 \\ \hline
5 & 26.780 & 0.07 & 13 & 0.07 \\ \hline
6 & 0.035 & 0.05 & 1 & 0 \\ \hline
7 & 0 & 0.04 & 5 & 0.02 \\ \hline
8 & 0 & 0.08 & 8 & 0.10 \\ \hline
9 & 0 & 0.07 & 25 & 2 \\ \hline
\end{tabu}
\end{center}

E de uma tabela com o preço de 100 gramas de cada tipo de alimento:
\begin{center}
\begin{tabu}{|c|c|c|c|c|c|c|c|c|c|}
\hline
Alimento & 1 & 2& 3 & 4 & 5 & 6 & 7 & 8 & 9\\ \hline
\rowfont{\color{red}} Preço & 0.60 & 1.00 & 5.00 & 1.00 & 0.50 & 0.20 & 0.15 & 0.40 & 1.00 \\ \hline

\end{tabu}
\end{center}
\end{frame}



\begin{frame}
 Cada alimento é uma \dt{variável de controle}, ou seja, 
 
% a variável $x_1$ representa a quantidade (em ``pacotes'' de 100g) do alimento 1 que a empresa deve comprar diariamente para cada funcionário, a variável $x_2$ representa a quantidade (em ``pacotes'' de 100g) do alimento 2 que a empresa deve comprar diariamente para cada funcionário, e assim por diante. Resumintdo, 
 
 \begin{center}
 $x_i$ representa a quantidade (em ``pacotes'' de 100g) do alimento $i$ que a empresa deve comprar diariamente para cada funcionário, com $i=1,\ldots,9$. 
 \end{center}\bigskip

A \dt{função-objetivo} que fornece o custo (que queremos minimizar) associado à compra  dos diversos tipos de alimentos para cada funcionário é dada por 
\begin{multline*}
C(x_1,x_2,x_3,x_4,x_5,x_6,x_7,x_8,x_9)= \\ {\color{red}0.6}x_1 + {\color{red} 1}x_2 +{\color{red} 5}x_3+{\color{red} 1}x_4+{\color{red} 0.5}x_5+{\color{red} 0.2}   x_6+{\color{red} 0.15}x_7+{\color{red} 0.4}x_8+{\color{red} 1}x_9.
\end{multline*}


\end{frame}



\begin{frame}
\frametitle{ }

Consultando as outras tabelas, podemos ver que o problema está sujeito às restrições impostas pelas quantidades mínimas de cada vitamina, a saber

\begin{small}
\begin{align*}
&0.14x_1+0.58x_2+0.15x_3+26.79x_5+0.035x_6\geq {\color{blue} 3},\\
&0.08x_1+1.3x_3+0.08x_4+0.07x_5+0.05x_6+0.04x_7+0.08x_8+0.07x_9\geq {\color{blue} 1.1},\\
&26x_3+38x_4+13x_5+x_6+5x_7+8x_8+25x_9\geq {\color{blue} 60},\\
&1.6x_1+6.9x_3+0.2x_4+0.07x_5+0.02x_7+0.1x_8+2x_9\geq {\color{blue} 11},\\
&x_i\geq {\color{blue} 0}, \ \forall i=1,\ldots,9.
\end{align*} 
\end{small}

\uncover<1->{Usando as técnicas que serão vistas neste curso de cálculo 3 é possível resolver este problema e obter a seguinte solução:
$$x_1=6.24,\  x_3=0.112, \ x_5=0.078,\ x_7=11.209$$
$$ x_2=x_4=x_6=x_8=x_9=0$$ }

\end{frame}

\begin{frame}
	\begin{casa}
		Revise as seções 12.5 e 12.6 do \cite{Stewart} e faça seguintes os exercícios
		\begin{enumerate}
			\item Determine as equações paramétricas da reta que passa pelos pontos $A=(1,3,2)$ e $B=(-4,3,0)$
			
			\item Determine a equação do plano que passa pelo ponto $A=(6,3,2)$ e é perpendicular ao vetor $\vec{n}=(-2,1,5)$
				
		\item Identifique e esboce as superfícies:
		\begin{enumerate}[a]
			\item $x^2+y^2+z^2=4$
			\item $z^2=y^2+4x^2$
			\item $z=x^2+y^2$
		\end{enumerate}
		
		\end{enumerate}
	\end{casa}
\end{frame}