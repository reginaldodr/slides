\section{Inversão de Matrizes}

\begin{frame}[label=inversao]{Matriz Inversa}

\begin{defin}
Dizemos que uma matriz quadrada $A$, de ordem $n$, é {\color{blue} invertível} ou {\color{blue} não singular}, se existe uma matriz $B$, também de ordem $n$ tal que
\[AB=BA=I_n,\]
em que $I_n$ é a matriz identidade. A matriz $B$ é chamada {\color{blue}inversa de $A$} e é denotada por $A^{-1}$. Se $A$ não tem inversa, dizemos que $A$ {\color{blue} não é invertível} ou é {\color{blue}{singular}}.
\end{defin}

\begin{exe}
A inversa da matriz 
$A=\begin{bmatrix}
-2 & 6\\ 0 & 1
\end{bmatrix}$ é a matriz $A^{-1}=\begin{bmatrix}
 -1/2 & 3\\ 0 & 1
\end{bmatrix}$
\end{exe}
\end{frame}

\subsection*{método de inversão}
\begin{frame}[label=inversao]

\begin{prop}
Para saber se $A$ é invertível, basta verificar uma das identidades:
\[AB=I_n \text{ ou } BA=I_n.\]
\end{prop}

\begin{teo}
Uma matriz é invertível se, e somente se, é linha-equivalente à matriz identidade.
\end{teo}

\begin{exe}
Verifique se a seguinte matriz é invertível
\[
A=\begin{bmatrix}
1 & 1 & 0\\ 0 & 2 & 1 \\ 1& 0 & 1
\end{bmatrix}
\] 
\end{exe}

\end{frame}


\begin{frame}[label=inversao]


\begin{exer}
Verifique se a seguinte matriz é invertível
\[
A=\begin{bmatrix}
1 & 2 & 3\\ 1 & 1 & 2 \\ 0& 1 & 2
\end{bmatrix}
\] 
\end{exer}

\end{frame}


\begin{frame}[label=inversao]


\begin{casa}
Verifique se as seguintes matrizes são invertíveis e em caso positivo, calcule a inversa:
\[
A=\begin{bmatrix}
1 & 2 & 3\\ 1 & 1 & 2 \\ 0 & 1 & 1  
\end{bmatrix},
B=\begin{bmatrix}
1 & 1 & 1 & 1\\ 1 & 2 & -1 & 2 \\ 1 & -1 & 2 &  1\\ 1& 3 & 3& 2 
\end{bmatrix}
\] 
\end{casa}

\end{frame}

\begin{frame}[label=inversao]
\begin{block}{Propriedades}
\begin{enumerate}
\item A inversa, quando existe, é única.

\item Se $A$ é invertível, então sua inversa também o é e vale $(A^{-1})^{-1}=A$.

\item Se $A$ e $B$ são invertíveis, então $AB$ é invertível e
\[(AB)^{-1}=B^{-1}A^{-1}. \]

\item Se $A$ é invertível, então $A^t$ também o é e vale
\[(A^t)^{-1}=(A^{-1})^t.\]

\item Para saber se $A$ é invertível, basta verificar uma das identidades:
\[AB=I_n \text{ ou } BA=I_n.\]

\item Um sistema $AX=B$ tem solução única se, e somente se, $A$ é invertível. Neste caso, a solução é $X=A^{-1}B$.
\end{enumerate}

\end{block}

\end{frame}

\subsection*{Invertendo matrizes com o sympy}

\begin{frame}[label=inversao,fragile=singleslide]{Invertendo matrizes com o sympy}
	\begin{footnotesize}
\begin{pyverbatim}
import sympy as sp

A=sp.Matrix([[1, 2,3],[1,1,3],[0,1,2]])
B=sp.Matrix([[1,1,1,1],[1,2,-1,2],[1,-1,2,1],[1,3,3,2]])
invA=A.inv()
invB=B.inv()
\end{pyverbatim}
	\end{footnotesize}		
\begin{pycode}
import sympy as sp

A=sp.Matrix([[1, 2,3],[1,1,3],[0,1,2]])
B=sp.Matrix([[1,1,1,1],[1,2,-1,2],[1,-1,2,1],[1,3,3,2]])
invA=A.inv()
invB=B.inv()
\end{pycode}
\[A=\pyl{A},\ B=\pyl{B}\]

\[A^{-1}=\pyl{invA},\ B^{-1}=\pyl{invB}\]
\end{frame}