\section{Matrizes}

\subsection*{Matrizes}

\begin{frame}[label=matrizes]{Matrizes}

Uma {\color{blue} matriz} $A$, de tamanho $m\times n$, é uma tabela de $mn$ números dispostos em $m$ linhas e $n$ colunas, e será representada como:
\[
A=(a_{ij})_{m\times n}=\left[\begin{array}{cccc}
 a_{11}  & a_{12}  & \cdots &  a_{1n}  \\
 a_{21} & a_{22}  & \cdots  & a_{2n} \\
 \vdots & \vdots & \cdots & \vdots \\
 a_{m1} & a_{m2}  & \cdots  & a_{mn} \\
\end{array}\right]
\]

Dizemos que $a_{ij}$ é o {\color{blue} elemento} ou a {\color{blue} entrada} de posição $i,j$ da matriz $A$. Denotamos por $\mathbb{M}_{m\times n}(\R)$ o conjunto de todas as matrizes $m\times n$ cujos elementos são números reais.
\medskip

Quando $m=n$, dizemos que $A$ é uma {\color{blue} matriz quadrada de ordem n}. Neste caso, os elementos da $a_{11}, a_{22}, \ldots, a_{nn}$ formam {\color{blue}diagornal principal} de $A$.

\end{frame}

\begin{frame}[label=matrizes,fragile=singleslide]{Matrizes}
\begin{pycode}
import sympy as sp

A=sp.Matrix([[0, -1],[1, 0]])
B=sp.Matrix([[1,3,0],[2,-1,-3]])
\end{pycode}


\begin{exe} São exemplos de matrizes:
\[A=\py{sp.latex(A)},\ B=\py{sp.latex(B)}.\] 
\end{exe}


\end{frame}

\subsection*{Operações com matrizes}


\begin{frame}[label=matrizes,fragile=singleslide]{Soma de Matrizes}
\begin{pycode}
import sympy as sp

A=sp.Matrix([[1, 2,-3],[3, 4,0]])
B=sp.Matrix([[-2,1,5],[0,3,-4]])
C=A+B
\end{pycode}

A {\color{blue} soma} de duas matrizes de {\color{blue} mesmo tamanho} $A=(a_{ij})_{m\times n}$ e $B=(b_{ij})_{m\times n}$ é definida como sendo a matriz $m\times n$ $C=A+B$, obtida somando-se os elementos correspondentes de $A$ e $B$, ou seja, 
\[c_{ij}=a_{ij}+b_{ij},\]
para cada $i=1,\ldots,m$ e $j=1,\ldots,n$.

\begin{exe}
A soma das matrizes \[
A=\py{sp.latex(A)} \text{ e } \ 
B=\py{sp.latex(B)}
\]
é a matriz 
\[C=\py{sp.latex(C)}\]
\end{exe}
\end{frame}


\begin{frame}[label=matrizes,fragile=singleslide]{Multiplicação por escalar}
\begin{pycode}
import sympy as sp

A=sp.Matrix([[-2,1],[0,3],[5,-4]])
a=-3
B=a*A
\end{pycode}

A {\color{blue} multiplicação de uma matriz} $A=(a_{ij})_{m\times n}$  {\color{blue} por um escalar} $\alpha\in \R$ é definida como sendo a matriz $m\times n$, $B=\alpha A$, obtida multiplicando-se cada elemento da matriz $A$ pelo escalar $\alpha$, isto é,
\[b_{ij}=\alpha a_{ij},\]
para cada $i=1,\ldots,m$ e $j=1,\ldots,n$.



\end{frame}

\begin{frame}[label=matrizes]

\begin{exe}
A multiplicação  da matriz $A=\py{sp.latex(A)} $ pelo escalar $\alpha=\py{sp.latex(a)}$ é a matriz
\[B=\py{sp.latex(a)} A=\py{sp.latex(B)}.\]
\end{exe}

\begin{block}{}
Com a soma e a multiplicação por escalar, {\color{blue}o conjunto das matrizes $\mathbb{M}_{m\times n}(\R)$ forma um espaço vetorial}!
\medskip

Em particular, o conjunto das  matrizes linha ou das matrizes coluna formam o espaço vetorial $\R^n$.
\medskip

De agora em diante, não usaremos mais a notação $\vec{v}$ para representar um vetor. Escreveremos simplesmente $v$ e no contexto ficará claro que se trata de um vetor de $\R^n$.
\end{block}

\end{frame}

\begin{frame}[label=matrizes]{Produto de Matrizes}

Dadas duas matrizes $A=(a_{ij})_{{\color{blue}m}\times {\color{red}p}}$ e $B=(b_{ij})_{{\color{red}p}\times {\color{blue}n}}$, definimos o {\color{blue} produto de $A$ por $B$} como sendo a matriz $C=(c_{ij})_{{\color{blue}m\times n}}$, definida por
\[c_{ij}=\sum_{{\color{red}k=1}}^{n} a_{i{\color{red}k}}b_{{\color{red}k}j}
=a_{i{\color{red}1}}b_{{\color{red}1}j} +a_{i{\color{red}2}}b_{{\color{red}2}j} +\cdots + a_{i{\color{red}n}}b_{{\color{red}n}j},\]
para cada $i=1,\ldots,m$ e $j=1,\ldots,n$.

\begin{small}
\[
\left[
\begin{tabu}{cccc}
 a_{11}  & a_{12}  & \cdots &  a_{1p} \\
 \vdots & \vdots & \cdots & \vdots\\
\rowfont{\color{red}}  a_{i1} & a_{i2}  & \cdots  & a_{ip}\\
 \vdots & \vdots & \cdots & \vdots  \\
 a_{m1} & a_{m2}  & \cdots  & a_{mp}
\end{tabu}
\right]
\left[
\begin{tabu}{ccccc}
 b_{11}  & \cdots  &{\color{red} b_{1j} }& \cdots &  b_{1n} \\
 b_{21}  & \cdots  & {\color{red} b_{2j} } & \cdots &  b_{2n} \\
 \vdots & \cdots & {\color{red}\vdots} & \cdots & \vdots\\
 b_{p1} & \cdots  & {\color{red} b_{pj} } & \cdots & b_{pn}
\end{tabu}
\right]
=
\left[
\begin{tabu}{ccc}
 c_{11}  &  \cdots &  c_{1n} \\
 \vdots &   {\color{red}c_{ij}} & \vdots\\
 c_{m1} & \cdots   &   c_{mn}
\end{tabu}
\right]
\]
\end{small}

\end{frame}

\begin{frame}[label=matrizes,fragile=singleslide]
\begin{pycode}
import sympy as sp

A=sp.Matrix([[1,2,-3],[3,4,0]])
B=sp.Matrix([[-2,1,0],[0,3,1],[5,-4,0]])
C=A*B
\end{pycode}

\begin{exe}
Considere as matrizes  $A=\py{sp.latex(A)} $ e $B=\py{sp.latex(B)} $. Então,
\[AB=\py{sp.latex(C)}.\]
\end{exe}
\end{frame}


\begin{frame}[label=matrizes]{Transformações Lineares}
Uma matriz $A$, de ordem $m\times n$, pode ser interpretada como uma {\color{blue}transformação} que leva vetores de $\R^n$ em vetores $\R^m$.

\begin{exer}
\begin{enumerate}
\item  A matriz 
$R=\begin{bmatrix}
0 & -1\\ 1 & 0 
\end{bmatrix}$ rotaciona qualquer vetor do plano $90^\circ$ no sentido anti-horário. Verifique esboçando no plano os vetores $v=\begin{bmatrix}
1\\ 1
\end{bmatrix}$ e $u=Rv$.

\item A matriz $S=\begin{bmatrix}
0 & 1\\ 1 & 0 
\end{bmatrix}$ reflete um vetor em relação à reta $y=x$. Verifique esboçando no plano os vetores $u$ do item anterior e o vetor  $w=Su$.
\end{enumerate}
\end{exer}


\end{frame}


\begin{frame}[label=matrizes]
Neste sentido, o {\color{blue}produto de matrizes} desempenha o papel de composição de funções.
\begin{exer}
Verifique isso calculando a matriz $M=SR$, onde $R$ e $S$ são as matrizes do exercício anterior e esboce os vetores $u$ e $Mu$.
\end{exer}


\end{frame}


\begin{frame}[label=matrizes,fragile=singleslide]
\begin{pycode}
import sympy as sp

R=sp.Matrix([[0,-1],[1,0]])
S=sp.Matrix([[0,1],[1,0]])
A=sp.Matrix([0,1])
B=sp.Matrix([1,1])
C=sp.Matrix([1,3])


\end{pycode}

\begin{casa}
Considere as matrizes  $R=\py{sp.latex(R)} $ e $S=\py{sp.latex(S)} $, e os vetores $u=\pyl{A}$, $v=\pyl{B}$ e $w=\pyl{C}$
\begin{enumerate}

\item Esboce o triângulo $ABC$ que tem como vértices as extremidades dos vetores.

\item Calcule $u'=Ru$, $v'=Rv$ e $w'=Rw$. Esboce o novo  triângulo $A'B'C'$ com vértices dados pelos novos vetores.

\item Calcule $u''=Su'$, $v''=Sv'$ e $w''=Sw'$. Esboce o triângulo $A''B''C''$ com vértices dados pelos novos vetores.

\item Calcule $M=SR$. Esboce o triângulo com vértices em $Mu$, $Mv$ e $Mw$.
\end{enumerate}
\end{casa}
\end{frame}





\begin{frame}[label=Matrizes]
\begin{casa}
Sejam 
$A=\begin{bmatrix}
1 & 2 \\
3 & 4
\end{bmatrix} $ e 
$B=\begin{bmatrix}
-2 & 1 \\
0 & 3
\end{bmatrix} $.
Calcule $AB$ e $BA$. 
\end{casa}
\end{frame}


\subsection*{Propriedades das Matrizes }

\begin{frame}[label=matrizes]{Propriedades das operações com Matrizes}

Sejam $A, B$ e $C$ matrizes com tamanhos apropriados e $\alpha,\beta\in \R$. São válidas as seguintes propriedades:

\begin{enumerate}
\item $A+B=B+A$ (comutatividade da soma)
\item $A+(B+C)=(A+B)+C$ (associatividade da soma)
\item $\alpha(\beta A)=(\alpha \beta)A$ (associatividade do produto por escalar)
\item $(\alpha+\beta)A=\alpha A+\beta A$ (distributividade do produto por escalar)
\item $\alpha(A+B)=\alpha A+\alpha B$ (distributividade do produto por escalar)
\item $A(BC)=(AB)C$ (associatividade do produto)
\item $A(B+C)=AB+AC$ e $(B+C)A=BA+CA$ (distributividade)
\end{enumerate}


\end{frame}


\begin{frame}
\begin{exe}
Se $A$ e $B$ são matrizes quadradas, então vale a identidade? \[(A+B)(A-B)=A^2-B^2.\]
\end{exe}
\end{frame}


\subsection*{Matriz Identidade}
\begin{frame}[label=matrizes]{Matriz Identidade}

A matriz $n\times n$, definida por
\[I_n=\begin{bmatrix}
1 & 0 & \cdots & 0\\
0 & 1 & \cdots & 0\\
\vdots & \vdots & \ddots & \vdots\\
0 & 0 & \cdots & 1
\end{bmatrix},\]
chamada {\color{blue} matriz identidade} é o \textbf{elemento neutro da multiplicação}, isto é, 
\[AI_n=I_mA=A,\]
para toda matriz $A=(a_{ij})_{m\times n}$.



\end{frame}





\subsection*{Transposta}
\begin{frame}[label=matrizes,fragile=singleslide]{Matriz Transposta}

A {\color{blue} transposta} de uma matriz $A=(a_{ij})_{{\color{blue}m}\times {\color{red}n}}$, denotada por $A^t$,  é a matriz obtida a partir de $A$ trocando-se as linhas com as colunas, isto é,  $A^t=(b_{ij})_{{\color{red}n}\times {\color{blue}m}}$, onde
\[b_{ij}=a_{ji},\]
para cada $i=1,\ldots,m$ e $j=1,\ldots,p$.


\begin{pycode}
import sympy as sp

A=sp.Matrix([[1,2,-3],[3,4,0]])
B=A.T
\end{pycode}

\begin{exe}
A transposta da matriz  $A=\py{sp.latex(A)} $ é a matriz 
\[A^t=\py{sp.latex(B)}.\]
\end{exe}

\end{frame}




\begin{frame}[label=matrizes,fragile=singleslide]{Propriedades da Transposta}

Sejam $A$ e $B$ matrizes com tamanhos apropriados e $\alpha,\in \R$. São válidas as seguintes propriedades:

\begin{enumerate}
\item $(A^t)^t=A$ 
\item $(A+B)^t=A^t+B^t$
\item $(\alpha A)^t=\alpha A^t$
\item $(AB)^t=B^tA^t$
\end{enumerate}

\end{frame}

\begin{frame}
\begin{casa}
\begin{enumerate}
\item Sejam    $A=\begin{bmatrix} x & 4 &  -2 \end{bmatrix}$  e  $B=\begin{bmatrix} 2 & -3 & 5 \end{bmatrix}$. Encontre o valor de $x$ tal que $AB^t=0,$ onde $0$ é a matriz nula, isto é, com todas as entradas sendo zero.

\item Calcule $M^3$, onde 
\[M=\begin{bmatrix}
0 & 1 & 0\\
0 & 0 & 1\\
0 & 0 & 0
\end{bmatrix}.
\]
\end{enumerate}
\end{casa}
\end{frame}

\subsection*{Usando o sympy para operar matrizes}



\begin{frame}[label=matrizes,fragile=singleslide]{Operações com matrizes usando o sympy}
	\begin{footnotesize}
\begin{pyverbatim}
import sympy as sp

A=sp.Matrix([[1, 2,-3],[3, 4,0]])
B=sp.Matrix([[-2,1,5],[0,3,-4]])
C=sp.Matrix([[-2,1,0],[0,3,0],[5,-4,0]])
S=A+B
display(S)
P=A*C
display(P)
T=A.T
display(T)

\end{pyverbatim}
	\end{footnotesize}		
\begin{pycode}
import sympy as sp

A=sp.Matrix([[1, 2,-3],[3, 4,0]])
B=sp.Matrix([[-2,1,5],[0,3,-4]])
C=sp.Matrix([[-2,1,0],[0,3,1],[5,-4,0]])
S=A+B
P=A*C
T=A.T
\end{pycode}
\[A=\pyl{A},\ B=\pyl{B},\ C=\pyl{C}\]

\[A+B=\pyl{S},\ AC=\pyl{P},\ A^t=\pyl{T}\]

\end{frame}





