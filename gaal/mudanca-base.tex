


\subsection*{Coordenadas em relação a uma base de $\R^n$}
\begin{frame}[label=lild]{Coordenadas em relação a uma base}
Se ${\color{red}\mathcal{B}=\{\vec{v}_1,\vec{v}_2,\ldots,\vec{v}_n\}}$ é uma base de $\R^n$, então cada vetor $\vec{v}\in \R^n$ pode ser escrito de forma única como
\[\vec{v}={\color{blue}c_1}{\color{red}\vec{v}_1}+{\color{blue}c_2}{\color{red}\vec{v}_2}+\cdots+{\color{blue}c_n}{\color{red}\vec{v}_n}.\]
Neste caso, dizemos que esta é a {\color{blue}expressão de $\vec{v}$ em termos da base $\mathcal{B}$}. Os escalares {\color{blue}$c_1,c_2,\ldots,c_n$} são chamados de {\color{blue}coordenadas} de $\vec{v}$ em relação à base $\mathcal{B}$ e escrevemos 
\[
[\vec{v}]_{{\color{red}\mathcal{B}}}=
{\color{blue}(c_1,c_2,\ldots,c_n)}_{{\color{red}\mathcal{B}}} \text{ ou }
[\vec{v}]_{{\color{red}\mathcal{B}}}=
{\color{blue}
\begin{bmatrix}
c_1 \\ c_2 \\ \vdots \\ c_n
\end{bmatrix}_{{\color{red}\mathcal{B}}}},
\]
chamado {\color{blue}vetor de coordenadas em relação à base $\mathcal{B}$}.

\end{frame}

\begin{frame}[label=lild]{}
\begin{exe}
\begin{enumerate}
%\item Determine as coordenadas do vetor $\vec{v}=(-1,2,1)$ em relação à base $\mathcal{B}=\{(1,0,1),(0,1,1)\}$ do espaço gerado por $\mathcal{B}$.

\item Determine as coordenadas do vetor $\vec{v}=(2,-1,3)$ em relação à base canônica do $\R^3$, $\mathcal{C}=\{\vec{e}_1,\vec{e}_2,\vec{e}_3\}$.

\item Mostre que $\mathcal{B}=\{(1,2,1),(2,9,0),(3,3,4)\}$ é uma base de $\R^3$. Em seguida, encontre o vetor de coordenadas de $\vec{v}=(5,-1,9)$ em relação a essa base.
\end{enumerate}
\end{exe}
\end{frame}