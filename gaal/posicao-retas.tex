\subsection*{Posição Relativa entre retas}

\begin{frame}[label=pos-retas]{Posição relativas entre retas}
	
\only<1,3>{	
\only<1>{No espaço, existem \textcolor{red}{quatro} posições que duas retas $r$ e $s$ podem assumir:
	\bigskip}

\begin{minipage}{0.45\textwidth}
	\begin{center}
		%\textbf{Concorrentes}
		\tdplotsetmaincoords{70}{80}
		\begin{tikzpicture}[tdplot_main_coords]
		
		\node at (0,0,3.5)	{\textbf{Paralelas}};
		
		
		\coordinate (O) at (0, 0, 0);
		
		\filldraw[gray, opacity = 0.5] (-1.5,-2,2) -- (1.5,-2,2) -- (1.5, 2, 2) --(-1.5, 2, 2) -- cycle;
		
		\draw[red,thick]  ($(-1,2,2)-0.2*(2,1,0)$) node[anchor=south] {$s$} -- ($(1,-2,2)-0.2*(2,1,0)$);
	
		\draw[blue,thick] ($(-1,2,2)+0.2*(2,1,0)$) node[anchor=north] {$r$} -- ($(1,-2,2)+0.2*(2,1,0)$);

\only<3>{
\draw plot [mark=*, mark size=1]  coordinates{($(-1,2,2)-0.2*(2,1,0)+0.3*(2,-4,0)$)} node[anchor=south] {{\color{red}$B$}}; 

\draw plot [mark=*, mark size=1] coordinates{($(-1,2,2)+0.2*(2,1,0)+0.7*(2,-4,0)$)} node[anchor=north] {{\color{blue}$A$}}; 

\draw[-{Latex},thick]  ($(-1,2,2)+0.2*(2,1,0)+0.7*(2,-4,0)$) -- ($(-1,2,2)-0.2*(2,1,0)+0.3*(2,-4,0)$);
}
\end{tikzpicture}
\end{center}
\end{minipage}
\begin{minipage}{0.45\textwidth}
	\begin{center}
		\tdplotsetmaincoords{70}{80}
	\begin{tikzpicture}[tdplot_main_coords]
	
	\node at (0,0,3.5)	{\textbf{Coincidentes}};
	
	
	\coordinate (O) at (0, 0, 0);
	
	\filldraw[gray, opacity = 0.5] (-1.5,-2,2) -- (1.5,-2,2) -- (1.5, 2, 2) --(-1.5, 2, 2) -- cycle;
	
	\draw[red,thick] (-1,2,2) node[anchor=south] {$s$} -- (1,-2,2) ;
	
	\draw[blue,thick] ($(-1,2,2)+0.02*(2,1,0)$) node[anchor=north] {$r$} -- ($(1,-2,2)+0.02*(2,1,0)$);
	
	
	\end{tikzpicture}		
	\end{center}
\end{minipage}

\begin{center}
	\tdplotsetmaincoords{70}{80}
\begin{tikzpicture}[tdplot_main_coords]

\draw[red,thick,-latex] ($(0,0,0)$)  -- ($(-1,2,0)$) node[anchor=south] {$\vec{s}$};
\draw[blue,thick,-latex] ($(0,0,0)$) -- ($0.7*(-1,2,0)$) node[anchor=north east] {$\vec{r}$};

\only<3>{\draw[thick,-latex] ($(0,0,0)$) -- ($-0.4*(2,1,0)-0.4*(2,-4,0)$) node[anchor=south east] {$\overrightarrow{{\color{blue}A}{\color{red}B}}$};
}
\end{tikzpicture}	

{\small {\color{blue}$\vec{r}$},{\color{red}$\vec{s}$} são \textcolor{blue}{LD} } 

\end{center}
\only<3>{Se ${\color{blue}\vec{r}}$ e $\overrightarrow{{\color{blue}A}{\color{red}B}}$ são {\color{blue} LI}, então elas são {\color{blue} paralelas}. Caso contrário, são {\color{red} coincidentes}.}
}


\only<2,4>{	
	\begin{minipage}{0.45\textwidth}
		\begin{center}
				\tdplotsetmaincoords{70}{80}
			\begin{tikzpicture}[tdplot_main_coords]
			
			\node at (0,0,3.5)	{\textbf{Concorrentes}};
			
			
			\coordinate (O) at (0, 0, 0);
			

			
			\draw[red,thick] (-1,2,2) node[anchor=south] {$s$} -- (1,-2,2);
			\filldraw[gray, opacity = 0.5] (-1.5,-2,2) -- (1.5,-2,2) -- (1.5, 2, 2) --(-1.5, 2, 2) -- cycle;
			\draw[blue,thick] (-1.5,-1.5,2)  -- (1.5,1.5,2) node[anchor=south] {$r$};

			
		\end{tikzpicture}
		\end{center}
	\end{minipage}
	\begin{minipage}{0.45\textwidth}
	\begin{center}
				\tdplotsetmaincoords{70}{80}
				\begin{tikzpicture}[tdplot_main_coords]
				
				\node at (0,0,3.5)	{\textbf{Reversas}};
				
				
				\coordinate (O) at (0, 0, 0);
				

				\filldraw[gray, opacity = 0.5] (-1.5,-2,0) -- (1.5,-2,0) -- (1.5, 2, 0) --
				(-1.5, 2, 0) -- cycle;				
				\draw[blue,thick] (-1.5,-1.5,0) -- (1.5,1.5,0)node[anchor=south] {$r$};
				
					\draw[red,thick] (-1,2,2) node[anchor=south] {$s$} -- (1,-2,2);
				\filldraw[gray, opacity = 0.5] (-1.5,-2,2) -- (1.5,-2,2) -- (1.5, 2, 2) --(-1.5, 2, 2) -- cycle;
				
				\draw[dashed] (-1.5,-2,0) -- (-1.5,-2,2) ;
		
\only<4>{
\draw plot [mark=*, mark size=1]  coordinates{($(1,-2,2)-0.7*(1,-2,2)+0.7*(-1,2,2)$)} node[anchor=south] {{\color{red}$B$}}; 

\draw plot [mark=*, mark size=1] coordinates{($(-1.5,-1.5,0)+0.3*(1.5,1.5,0)-0.3*(-1.5,-1.5,0)$)} node[anchor=north] {{\color{blue}$A$}}; 

\draw[-{Latex},thick]  ($(-1.5,-1.5,0)+0.3*(1.5,1.5,0)-0.3*(-1.5,-1.5,0)$) -- ($(1,-2,2)-0.7*(1,-2,2)+0.7*(-1,2,2)$);
}		
\end{tikzpicture}
\end{center}
\end{minipage}


\begin{center}
\only<2>{	\tdplotsetmaincoords{70}{80}
	\begin{tikzpicture}[tdplot_main_coords]
	
	\draw[red,thick,-latex] ($(0,0,0)$)  -- ($0.7*(-1,2,0)$) node[anchor=south] {$\vec{s}$};
	\draw[blue,thick,-latex] ($(0,0,0)$) -- ($(1,1,0)$) node[anchor=north east] {$\vec{r}$};
\only<4>{\draw[-{Latex},thick]  (0,0,0) -- ($-1*(-1.5,-1.5,0)-0.3*(1.5,1.5,0)+0.3*(-1.5,-1.5,0)   +(1,-2,2)-0.7*(1,-2,2)+0.7*(-1,2,2)$);}

	\end{tikzpicture}	

{\small {\color{blue}$\vec{r}$},{\color{red}$\vec{s}$} são \textcolor{blue}{LI} } }
\end{center}

\only<4>{
\begin{minipage}{0.6\textwidth}
Se ${\color{blue}\vec{r}}$, ${\color{red}\vec{s}}$ e $\overrightarrow{{\color{blue}A}{\color{red}B}}$ são {\color{blue} LI}, então elas são {\color{blue} reversas}. Caso contrário, são {\color{red} concorrentes}.
\end{minipage}
\begin{minipage}{0.3\textwidth}
	\tdplotsetmaincoords{70}{80}
	\begin{tikzpicture}[tdplot_main_coords]
	
	\draw[red,thick,-latex] ($(0,0,0)$)  -- ($0.7*(-1,2,0)$) node[anchor=south] {$\vec{s}$};
	\draw[blue,thick,-latex] ($(0,0,0)$) -- ($(1,1,0)$) node[anchor=north east] {$\vec{r}$};
\only<4>{\draw[-{Latex},thick]  (0,0,0) -- ($-1*(-1.5,-1.5,0)-0.3*(1.5,1.5,0)+0.3*(-1.5,-1.5,0)   +(1,-2,2)-0.7*(1,-2,2)+0.7*(-1,2,2)$);}

	\end{tikzpicture}	

{\small {\color{blue}$\vec{r}$},{\color{red}$\vec{s}$} são \textcolor{blue}{LI}}
\end{minipage}
}}
\end{frame}


\begin{frame}[label=pos-retas]{Posição relativas entre retas}

\begin{exe}
Estude a posição relativa das retas:
\begin{enumerate}
\item $r: X=(1,2,3)+t(0,1,3)$ e $s: X=(0,1,0)+t(1,1,1)$, $t\in \R.$

\item $r: X=(1,2,3)+t(0,1,3)$ e $s: X=(1,3,6)+t(0,2,6)$, $t\in \R.$

\item $r: X=(1,2,3)+t(0,1,3)$, $t\in \R$ e 
$s: \begin{cases}
x+y+z=6\\
x-y-z=-4
\end{cases}$
\end{enumerate}
\end{exe}

\end{frame}
