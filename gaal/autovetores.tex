\section{Autovetores e Autovalores}

\subsection*{Autovetores}

\begin{frame}[label=autovalores]{Autovetores e Autovalores}

\begin{defin}
Seja $A$ uma matriz quadrada de ordem $n$. Dizemos que um vetor {\color{red}não nulo} $\vec{v}\in \R^n$ é um  {\color{blue}autovetor} de $A$ se $A\vec{v}$ for um múltiplo escalar de $\vec{v}$, isto é, se existe  $\lambda \in \R$ tal que 
\[A\vec{v}=\lambda \vec{v}.\]
Neste caso, dizemos que $\lambda$ é um {\color{blue}autovalor} de $A$.

\end{defin}

\begin{exe}
O vetor $\vec{v}=\begin{bmatrix} 1\\2 \end{bmatrix}$ é um autovetor da matriz
$A=\begin{bmatrix}
3 & 0 \\ 8 & -1
\end{bmatrix}
$
\end{exe}

\end{frame}

\begin{frame}[label=autovalores]{}
\begin{teo}
$\lambda\in \R$ é um autovalor de uma matriz $A$ se, e só se, 
\[\det(A-\lambda I)=0.\]
Está equação e dita {\color{blue}equação característica de A}. E o polinômio $p(\lambda)=\det(A-\lambda I)$ é denominado {\color{blue}polinômio característico}.
\end{teo}

\begin{exe}
Determine os autovetores e autovalores da matriz
\[A=
\begin{bmatrix}
1 & -1 \\ -4 & 1
\end{bmatrix}
\]
\end{exe}
\end{frame}


\subsection*{Autoespaço}
\begin{frame}[label=autovalores]{Autoespaço}
Os autovetores associados a um mesmo autovalor $\lambda$ são os vetores não nulos  que satisfazem
\[(A-\lambda I)\vec{x}=\vec{0},\]
isto é, {\color{red}são todos os vetores não nulos} pertencentes ao {\color{red}núcleo de $A-\lambda I$}, chamado de {\color{blue}autoespaço de $A$ associado a $\lambda$} e denotado por $E_{\lambda}$. 


\begin{exe}
Determine uma base dos autoespaços da  matriz 
\[A=\begin{bmatrix}
0 & 0 & -2\\ 1 & 2 & 1 \\ 1 &0& 3
\end{bmatrix}. \]
\end{exe}
\end{frame}

\begin{frame}[label=autovalores,fragile=singleslide]{}
\begin{casa}
\begin{pycode}
import sympy as sp
A=sp.Matrix([[0,0,2,0],[1,0,1,0],[0,1,-2,0],[0,0,0,1]])
\end{pycode}
Considere a matriz
\[A=
\pyl{A}
\]
\begin{enumerate}
\item Determine os autovalores de $A$.
\item  Determine os autoespaços de $A$.
\item  Determine uma base para cada autoespaço e a dimensão.
\end{enumerate}
\end{casa}
\end{frame}


\subsection*{Matrizes Semelhantes}
\begin{frame}[label=autovalores]{Matrizes Semelhantes}
\begin{defin}
Dizemos que uma {\color{red}matriz quadrada} $B$ é {\color{blue}semelhante} a um matriz $A$ quando existir alguma matriz invertível $P$ tal que 
\[B=P^{-1}AP.\]
Neste caso, escrevemos $B\sim A$.
\end{defin}

Matrizes semelhantes têm muitas propriedades em comum, por exemplo, 

\begin{enumerate}
\item Matrizes semelhantes têm o mesmo determinante.
\item Se $A$ e $B$ são semelhantes, então $B^k=P^{-1}A^kP$.
\end{enumerate}


%\begin{center}
%{\color{red}Matrizes semelhantes têm o mesmo determinante.}
%\end{center}



\end{frame}

\begin{frame}[label=autovalores]{}
\begin{exe}
Defina $P$ como a matriz cujas colunas são os autovetores da matriz $A=\begin{bmatrix}
0 & 0 & -2\\ 1 & 2 & 1 \\ 1 &0& 3
\end{bmatrix}$ do exemplo anterior e calcule a matriz  $D=P^{-1}AP$, semelhante a $A$.
\end{exe}

\begin{teo}
Sejam $\lambda_1$ e $\lambda_2$ {\color{red}autovalores distintos} de uma matriz $A$. Se $\mathcal{B}_1$ e $\mathcal{B}_2$ são as bases dos autoespaços $E_{\lambda_1}$ e $E_{\lambda_2}$, respectivamente, então $\mathcal{B}=\mathcal{B}_1\cup \mathcal{B}_2$ é LI.
\end{teo}

\end{frame}


\begin{frame}[label=autovalores]{Diagonalização de Matrizes}
\begin{defin}
Dizemos que uma matriz quadrada $A$ é {\color{blue}diagonalizável} se for semelhante a alguma matriz diagonal, isto é, se existe alguma matriz invertível $P$ tal que $D=P^{-1}AP$ é uma matriz diagonal. Neste caso, dizemos que $P$ {\color{blue}diagonaliza } $A$.
\end{defin}

\begin{teo}
Uma matriz $A$ de ordem $n$ é diagonalizável se, e só se, possui $n$ autovetores linearmente independentes. Neste caso, se $D=P^{-1}AP$ a matriz diagonal $D$ é formada pelos autovalores e a matriz $P$ tem as colunas formadas pelos autovetores.
\end{teo}

\end{frame}

\begin{frame}[label=autovalores,fragile=singleslide]{}
\begin{exe}
\begin{enumerate}
\item Mostre que a seguinte matriz não é diagonalizável
\[
A=\begin{bmatrix}
1& 0 & 0& \\ 1 & 2 & 0\\ -3 & 5 &  2
\end{bmatrix}
\]

\item Calcule $A^n$ onde $A=\begin{bmatrix}
0 & 1 \\ 2 & 1
\end{bmatrix}$

\begin{pycode}
import sympy as sp

A=sp.Matrix([[4,0,1],[2,3,2],[1,0,4]])
\end{pycode}

\item Mostre que a matriz $A$ é diagonalizável e encontre as matrizes diagonal $D$ e a matriz $P$ que diagonaliza $A$.
\[
A=\pyl{A}
\]
\end{enumerate}

\end{exe}

\end{frame}

\begin{frame}[label=autovalores,fragile=singleslide]{}

\begin{casa}
\begin{pycode}
import sympy as sp

A=sp.Matrix([[4,2,2],[2,4,2],[2,2,4]])
B=sp.Matrix([[-1,7,-1],[0,1,0],[0,15,-2]])
\end{pycode}
\begin{enumerate}
\item Mostre que 
\[A=\pyl{A}\]
é diagonalizável e obtenha as matrizes diagonal $D$ e a matriz $P$ que diagonaliza $A$.

\item Calcule $B^{11}$, onde 
\[B=\pyl{B}\]
\end{enumerate}

\end{casa}
\end{frame}

\begin{frame}[label=autovalores,,fragile=singleslide]{Diagonizando com sympy}
	\begin{footnotesize}
\begin{pyverbatim}
import sympy as sp

A=sp.Matrix([[-1, 2, -2, 10],
[0, -7, 6, -30],
[0, 2, 0, 7],
[0, 2, -2, 9]])
[P,D]=A.diagonalize()
display(P)
display(D)
\end{pyverbatim}
	\end{footnotesize}		

\begin{pycode}
import sympy as sp

A=sp.Matrix([[-1, 2, -2, 10], [0, -7, 6, -30], [0, 2, 0, 7], [0, 2, -2, 9]])
[P,D]=A.diagonalize()
\end{pycode}
\[A=\pyl{A},\, P=\pyl{P},\]
\[D=\pyl{D}\]

\end{frame}

\begin{frame}[label=autovalores]{}


\end{frame}
