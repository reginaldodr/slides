\section{Determinantes}

\begin{frame}[label=determinantes]{Determinantes}

O {\color{blue}determinante} é uma função que a cada matriz quadrada $A$ associa um número real, denotado por {\color{blue} $\det(A)$}. Vamos definir o determinante de forma induzida.
\medskip

\begin{itemize}
\item {\color{blue} \textbf{Matrizes $1\times 1$}:}

\[\det(A)=\det([a_{11}])=a_{11}.\]

\item {\color{blue} \textbf{Matrizes $2\times 2$}:}

\[det(A)=\det \begin{bmatrix}
{\color{blue}a_{11}} &{\color{teal} a_{12}}\\ {\color{teal}a_{21}} & {\color{blue}a_{22}}
\end{bmatrix}={\color{blue}a_{11}a_{22}}{\color{red}-}{\color{teal}a_{12}a_{21}}.  \] 
\end{itemize}

\begin{exe}
\[\det\begin{bmatrix}
2 & 2\\ -1 & 20 
\end{bmatrix}=40{\color{red}-}(-2)=42\]
\end{exe}

\end{frame}

\begin{frame}[label=determinantes]
Se $A$ for uma matriz quadrada de ordem $n$, denotaremos por $A_{ij}$ a submatriz de $A$, de ordem $n-1$, obtida mediante a omissão da $i$-ésima linha e da $j$-ésima coluna de $A$.
\begin{defin}
Dada uma matriz quadrada $A$. Definimos o {\color{blue} menor do elemento $a_{ij}$} como $\det(A_{ij})$. E o {\color{blue} cofator do elemento $a_{ij}$} como 
\[c_{ij}=(-1)^{i+j}\det(A_{ij}).\]
\end{defin}
\begin{exe}
Seja $A=\begin{bmatrix}
1 & 3 & 0 \\ -2 & 1 & -1 \\ 2 & 0 & 1
\end{bmatrix}$. Então $A_{12}=\begin{bmatrix}
 -2  & -1 \\ 2 &  1 
\end{bmatrix}$. E o cofator do elemento $a_{12}$ é $c_{12}=-\det(A_{12})=-(-2-(-2))=0$.
\end{exe}

\end{frame}

\begin{frame}[label=determinantes]{Determinantes ordem $n>2$}
\begin{defin}
Seja $A$ uma matriz quadrada de ordem $n>2$. Definimos o {\color{blue}determinante} de $A$ indutivamente por:
\[\det(A)=\sum_{j=1}^{n}a_{1j}c_{1j}=\sum_{j=1}^{n}a_{1j}(-1)^{1+j}\det(A_{1j}).\]
Essa soma é chamada de {\color{blue}expansão em cofatores} do determinante de $A$.
\end{defin}

\begin{exe}
Calcule o determinate da matriz 
\[A=\begin{bmatrix}
1 & 3 & 0 \\ -2 & 1 & -1 \\ 2 & 0 & 1
\end{bmatrix}.\]
\end{exe}

\end{frame}


\begin{frame}[label=determinantes]
\begin{teo}
Seja $A$ uma matriz quadrada de ordem $n$. O determinante de $A$ pode ser calculado fazendo-se o desenvolvimento em cofatores segundo {\color{blue}} qualquer linha ou qualquer coluna.
\end{teo}

\begin{exe}
Calcule o determinante da matriz 
\[
A=\begin{bmatrix}
 1 & 0 & 3 & 1 \\ -1 & 3 & 2 & 5 \\ 	0 & 0 & 2  & 3\\ 2 & 1 & -2 & 0
\end{bmatrix}.
\]
\end{exe}

\end{frame}


\subsection*{Determinantes com o sympy}

\begin{frame}[label=determinantes,fragile=singleslide]{Determinantes com o sympy}
	\begin{footnotesize}
\begin{pyverbatim}
import sympy as sp

A=sp.Matrix([[1,0,3,1],[-1,3,2,5],[0,0,2,3],[2,1,-2,0]])
A.det()
\end{pyverbatim}
	\end{footnotesize}		
\begin{pycode}
import sympy as sp

A=sp.Matrix([[1,0,3,1],[-1,3,2,5],[0,0,2,3],[2,1,-2,0]])
detA=A.det()
\end{pycode}
\[\det(A)=\det \pyl{A}=\pyl{detA}\]
\end{frame}

\begin{frame}[label=determinantes]
\begin{block}{Um restaurante no fim do universo}
Para se calcular o determinante de uma matriz $n\times n$ pela expansão em cofatores, precisamos fazer $n$ produtos e calcular $n$ determinantes de matrizes $(n-1)\times (n-1)$, que por sua vez vai precisar calcular $n-1$ produtos e assim por diante. Portanto, ao todo são necessários da ordem de $n!$ produtos.
\medskip


Mesmo um supercomputador não pode calcular determinantes de matrizes moderadamente grandes usando cofatores!
\medskip


Para se calcular o determinante de uma matriz $50\times 50$, é necessário se realizar $50!\approx 3\times  10^{64}$ produtos. Um supercomputador pode realizar da ordem de $10^{17}$ (100 quadrilhões) operações por segundo. Portanto, precisaria de $3\times 10^{47}$ segundos para calcular esse determinante, isto é, aproximadamente $10^{39}$ anos!. A estimativa da idade do universo é de 10 bilhões de anos, isto é $10^{10}$.
\end{block}
\end{frame}

\subsection*{Propriedades dos Determinantes}


\begin{frame}[label=determinantes]{Propriedades dos determinantes}

O determinante é um {\color{blue}função $n$-linear} das linhas ou das colunas. Vamos precisar o que isto quer dizer.
\medskip

Podemos representar uma matriz $A=(a_{ij})_{n\times n}$  em termos de suas linhas, isto é, 
\[A=\begin{bmatrix}
A_1 \\ A_2 \\ \vdots\\ A_n
\end{bmatrix}, \]
em que $A_i$ é a $i$-ésima linha, ou seja, 
$A_i=\begin{bmatrix} a_{i1} & a_{i2} & \cdots & a_{in} \end{bmatrix}$. 


Suponha que uma linha $A_k$ é decomposta como uma combinação linear de dois vetores linhas, isto é, $A_k=\alpha X+\beta Y$, onde $\alpha,\beta \in \R$, 
$X=\begin{bmatrix} x_{1} & x_{2} & \cdots & x_{n} \end{bmatrix}$ e $Y=\begin{bmatrix} y_{1} & y_{2} & \cdots & y_{n} \end{bmatrix}$.



\end{frame}

\begin{frame}[label=determinantes]
Então, 
\[\det\begin{bmatrix}
A_1 \\ A_2 \\ \vdots\\ A_k \\ \vdots \\ A_n
\end{bmatrix}
=\det\begin{bmatrix}
A_1 \\ A_2 \\ \vdots\\ \alpha X+\beta Y\\ \vdots \\ A_n
\end{bmatrix}=\alpha\, \det\begin{bmatrix}
A_1 \\ A_2 \\ \vdots\\ X \\ \vdots \\ A_n
\end{bmatrix}+\beta\, \det\begin{bmatrix}
A_1 \\ A_2 \\ \vdots\\ Y\\ \vdots \\ A_n
\end{bmatrix}\]

 Esta propriedade também vale para as colunas.

\end{frame}

\begin{frame}[label=determinantes]


\begin{exe} Calcule o determinante
\[\det\begin{bmatrix}
\cos(t) & \sin(t)\\
2\cos(t)-3\sen(t) & 2\sin(t)+3\cos(t)
\end{bmatrix}
\]

\end{exe}
\end{frame}

\begin{frame}[label=determinantes]{Cálculo de Determinantes por Redução por linhas}

\begin{itemize}

\item Se uma matriz $A$ possui duas linhas ou duas colunas iguais, então 
\[\det(A)=0.\]

\item Se $B$ é obtida de $A$ multiplicando-se uma linha por um escalar $\alpha$, então 
\[\det(B)=\alpha \det(A).\]

\item O determinante é uma {\color{blue}função alternada}, isto é, Se $B$ resulta de $A$ pela troca da posição de duas linhas ou colunas, então 
\[\det(B)=-\det(A).\]

\item Se $B$ é obtida de $A$  substituindo-se uma linha pela soma dela com um múltiplo escalar de outra linha, então 
\[\det(B)=\det(A).\]

\end{itemize}

\end{frame}

\begin{frame}[label=determinantes,fragile=singleslide]
\begin{pycode}
import sympy as sp

A=sp.Matrix([[0,2,-4,5],[3,0,-3,6],[2,4,5,3],[5,-1,-3,1]])
\end{pycode}
\begin{itemize}
\item  O determinante de uma matriz triangular superior ou inferior é o produto dos elementos da diagonal principal, isto é,
\[
\det \begin{bmatrix}
a_{11} & 0 & \cdots & 0\\
a_{21} & a_{22} & \cdots & 0\\
\vdots & \vdots & \ddots & 0\\ 
a_{n1} & a_{n2} & \cdots & a_{nn}
\end{bmatrix}= a_{11}a_{22}\cdots a_{nn}.
\]
\end{itemize}

\begin{exe}
Calcule  o determinante da seguinte matriz transformando-a em uma matriz triangular superior:
\[A=\pyl{A}\]

Resposta: $\det(A)=\pyl{A.det()}$.
\end{exe}



\end{frame}





\begin{frame}[label=determinantes]{Propriedades dos Determinantes}

\begin{itemize}
\item $\det(AB)=\det(A)\det(B)$.

\item $\det(A^t)=\det(A)$.

\item $A$ é invertível se, e somente se $\det(A)\neq 0$. Neste caso, \[\det(A^{-1})=\frac{1}{\det(A)}.\]

\item O sistema homogêneo $AX=0$ tem solução não trivial se, e somente se, $\det(A)=0$.
\end{itemize}

\end{frame}

\begin{frame}[label=determinantes]

\begin{exe}Seja 
\[T=\begin{bmatrix}
2 & 2 & 2\\ 0 & 2 & 0 \\ 0 & 1 & 3
\end{bmatrix}.\]
Determine os valores de $\lambda \in \R$ para os quais o sistema $TX=\lambda X$ tem solução não trivial.
\end{exe}

\end{frame}


\begin{frame}[label=determinantes]


\begin{block}{Fórmula da Inversa para matrizes $2\times 2$}
Um matriz 
\[A=\begin{bmatrix}
a & b \\ c & d 
\end{bmatrix}\]
 é invertível se, e somente se, $\det(A)=ad-bc\neq 0$ e neste caso
\[A^{-1}=\frac{1}{\det(A)}\begin{bmatrix}
d & -b \\ -c & a 
\end{bmatrix}.\]

\end{block}

\end{frame}



