
\subsection*{Subespaço}
\begin{frame}[label=lild]{Subespaço Vetorial}
\begin{defin}
Um subconjunto {\color{red}não-vazio} $V\subset\R^n$ é chamado de {\color{blue} subespaço} de $\R^n$ quando satisfaz:
\begin{enumerate}
\item Se $\vec{u},\vec{v}\in V$, então $\vec{u}+\vec{v}\in V$,
\item Se $\vec{v}\in V$, então $\alpha\vec{v}\in V$, para todo $\alpha\in \R$.
\end{enumerate}
\end{defin}

\begin{exe}
Mostre que o conjunto $V=\{(x,y,z)\in \R^3;z=0\}$ é um subespaço de $\R^3$.
\end{exe}

\end{frame}

\begin{frame}[label=lild]{}
\begin{exe}
\begin{enumerate}
\item Os subespaços vetoriais triviais de $\R^n$ são $V=\{\vec{0}\}$ e o próprio $V=\R^n$.

\item Se $A$ é uma matriz $m\times n$, então o conjunto solução do sistema homogêneo $AX=\vec{0}$ é um subespaço de $\R^n$, chamado de {\color{blue}espaço solução} ou {\color{blue}espaço nulo} ou {\color{blue}núcleo de A} e será denotado por {\color{blue} $\ker(A)$}.

\item Retas e planos em $\R^3$, passando pela origem, são subespaços de $\R^3$.

\item Determine o subespaço de $\R^5$ que é solução do sistema 
\[\left\{
\begin{array}{rrrrrr}
x_1 &+ x_2 & & &+x_5&=0\\
-2x_1 & -2x_2 &+ x_3& -x_4 & -x_5& =0 \\
x_1&+ x_2& -x_3 &+ x_4& &=0
\end{array}\right.
\]
\end{enumerate}
\end{exe}


\end{frame}

\subsection*{Geradores}
\begin{frame}[label=lild]{}
\begin{defin}
Seja $V$ um subespaço de $\R^n$. Dizemos que os vetores $\vec{v}_1,\vec{v}_2,\ldots,\vec{v}_k$, petencentes a $V$, {\color{blue}geram} $V$, se qualquer vetor de $V$ pode ser escrito como combinação linear dos vetores $\vec{v}_1,\vec{v}_2,\ldots,\vec{v}_k$. Neste caso, também dizemos que $V$ é o {\color{blue}subespaço gerado por} $\vec{v}_1,\vec{v}_2,\ldots,\vec{v}_k$ e o denotaremos por $\operatorname{span}\{\vec{v}_1,\vec{v}_2,\ldots,\vec{v}_k\}$.
\end{defin}

\begin{exe}
\begin{enumerate}
\item Determine um conjunto de geradores do núcleo da matriz $A$ do exemplo anterior.

\item Mostre que $\vec{w}=(4,-1,8)$ não está em $\operatorname{span}\{(1,2,-1),(6,4,2)\}$.
\end{enumerate}
\end{exe}
%
%\begin{block}{}
%O espaço solução de um sistema homogêneo $AX=\vec{0}$ é gerado pelas linhas da matriz $A$, também chamado de {\color{blue}espaço linha}.
%\end{block}

\end{frame}

\subsection*{Base}
\begin{frame}[label=lild]{Base}
\begin{defin}
Seja $V$ um subespaço de $\R^n$. Dizemos que um subconjunto $\mathcal{B}=\{\vec{v}_1,\vec{v}_2,\ldots,\vec{v}_k\}$ de $V$ é uma {\color{blue}base} de $V$, se
\begin{enumerate}
\item $\mathcal{B}$ é um conjuntos de geradores de $V$.
\item $\mathcal{B}$ é LI.
\end{enumerate}
\end{defin}
\begin{exe}
\begin{enumerate}
\item O conjunto $\vec{e}_1=(1,0,\ldots,0), \vec{e}_2=(0,1,\ldots,0),\ldots, \vec{e}_n=(0,0,\ldots,1)$ é uma base do $\R^n$, chamada de {\color{blue}base canônica}.
\item Determine uma base para o plano $x+2y-z=0$.
\end{enumerate}

\end{exe}



\end{frame}

\begin{frame}[label=lild]{Dimensão}

\begin{teo}
Se $\mathcal{B}=\{\vec{v}_1,\vec{v}_2,\ldots,\vec{v}_k\}$ é uma base de $V\subset\R^n$, então qualquer outra base tem $k$ elementos. 
\end{teo}


\begin{defin}
A {\color{blue}dimensão} de um subespaço vetorial $V\subset\R^n$ é definido como o número de elementos de uma base de $V$.
\end{defin}

\begin{exe}
 Determine uma base e a dimensão do subespaço $V=\{(a+c,b+c,a+b+2c);\, a,b,c\in \R\}$.

\end{exe}

\end{frame}





\subsection*{Núcleo}
\begin{frame}[label=lild]{Núcleo de uma matriz}
Definimos anteriormente que o {\color{blue}núcleo} de uma matriz $A$ é o espaço solução do sistema $AX=\vec{0}$, denotado por {\color{blue}$\ker(A)$}. {\color{blue}As operações elementares com linhas não alteram o núcleo de uma matriz}. Pode-se mostrar que 
\[{\color{red}\dim(\ker(A))=\operatorname{nulidade}(A)}\]
isto é, o número de variáveis livres do sistema. 
%\[{\color{red} \dim(\ker(A))+\operatorname{posto}(A)=n},\]
%onde $n$ é número de colunas da matriz $A$. 
Vejamos, através do próximo exemplo, um procedimento para obter-se uma base para o núcleo de uma matriz.

\begin{exe}
Determine uma base para o núcleo da matriz
\[
A=\begin{bmatrix}
1 & 3 & -2 & 0 & 2 & 0\\
2 & 6 & -5 & -2& 4& -3& \\
0 & 0& 5& 10& 0 & 15\\
2 & 6& 0&  8& 4& 18
\end{bmatrix}
\]
\end{exe}


\end{frame}

\subsection*{Espaço linha e Espaço coluna}
\begin{frame}[label=lild]{Espaço linha e Espaço coluna}

Dada uma matriz $A$  $m\times n$, além do núcleo de $A$, ainda existem mais dois subespaços associados à ela.
\begin{enumerate}
\item {\color{blue}O espaço linha de $A$}: subespaço de $\R^n$ gerado pela linhas de $A$.
\item {\color{blue}O espaço coluna de $A$}: subespaço de $\R^m$ gerados pelas colunas de $A$.
\end{enumerate}

\begin{teo}
Se uma matriz {\color{red}$R$ está em forma escalonada por linhas}, então os vetores linha não nulos formam uma base do espaço linha de $R$, e os vetores coluna que contém os pivôs formam uma base do espaço coluna. 
\end{teo}

\begin{exampleblock}{}
Em particular, os espaços linha e coluna de $R$ têm a mesma dimensão, apesar de não serem  subespaços de espaços vetoriais diferentes!
\end{exampleblock}

\end{frame}


\begin{frame}[label=lild]{}
Note que a matriz a seguinte matriz está na forma escalonada
\[
R=\begin{bmatrix}
{\color{red}1} & -2 & 5 & 0 & 3\\
0& {\color{red}1}& 3 & 0 & 0 \\
0& 0& 0 & {\color{red}2} & 0\\
0& 0& 0 & 0 & 0
\end{bmatrix}.
\]
Então, 
\begin{itemize}
\item Uma base para o espaço linha de $R$ são os vetores: $\vec{r}_1=(1, -2, 5, 0, 3), \vec{r}_2=(0,1,3,0, 0), \vec{r}_3=(0,0,0,2,0)$ 

\item Uma base para o espaço coluna de $R$ são os vetores:
\[
\vec{c}_1=\begin{bmatrix}
1 \\ 0 \\ 0\\ 0
\end{bmatrix},\
\vec{c}_2=\begin{bmatrix}
-2 \\ 1 \\ 0\\ 0
\end{bmatrix},\
\vec{c}_3=\begin{bmatrix}
0 \\ 0 \\ 2\\ 0
\end{bmatrix}.
\]

\end{itemize}

\end{frame}



\begin{frame}[label=lild]{Obtendo base para os espaço linha}


{\color{blue}As operações elementares com linhas não alteram o espaço linha de uma matriz}, pois as novas linhas são combinações lineares das anteriores e as operações são reversíveis. Portanto, para obter uma base do espaço linha, basta escalonar a matriz.  

\begin{exe}
Determine uma base para o espaço linha da matriz
\[
A=\begin{bmatrix}
1 & -3 & 4 & -2 & 5 & 4\\
2 & -6 & 9 & -1 & 8 & 2\\
2 & -6 & 9 & -1 & 9 & 7\\
-1 & 3& -4 & 2 & -5 & -4
\end{bmatrix}.
\]
\end{exe}
\end{frame}


\begin{frame}[label=lild]{Obtendo base para os espaço coluna}
Diferentemente do espaço linha, as operações elementares com linhas podem alterar o espaço coluna. 

\begin{exe}
As seguintes matrizes são linha-equivalente, mas têm espaços coluna diferentes
\[
A=\begin{bmatrix}
1 & 3\\ 2& 6
\end{bmatrix},\ 
B=\begin{bmatrix}
1 & 3\\ 0 & 0
\end{bmatrix}.
\]
\end{exe}

\end{frame}


\begin{frame}[label=lild]{ }
Dadas duas matrizes ${\color{blue}A}$ e ${\color{red}B}$ linha-equivalentes, sabemos que:
\begin{enumerate}
\item  Os sistemas ${\color{blue}A}X=\vec{0}$ e ${\color{red}B}X=\vec{0}$ {\color{red}têm exatamente as mesmas soluções!}
\item Uma solução $X=\begin{bmatrix}
c_1\\ c_2 \\ \vdots \\ c_n
\end{bmatrix}$ desses sistemas nos dá uma combinação linear das colunas, isto é, se {\color{blue}$A_1, A_2, \ldots, A_n$} e {\color{red}$B_1, B_2,\ldots B_n$} são as respectivas colunas de {\color{blue}$A$} e {\color{red}$B$}, então
\[c_1{\color{blue}A_1}+c_2{\color{blue}A_2}+\cdots c_n{\color{blue}A_n}=\vec{0}\]
\[c_1{\color{red}B_1}+c_2{\color{red}B_2}+\cdots c_n{\color{red}B_n}=\vec{0}\]
\end{enumerate}


Portanto, {\color{blue} as colunas de $A$ são LI se e só se, as colunas de $B$ são LI}. Em outras palavras, {\color{red} ao escalonarmos uma matriz, as colunas mudam mas a dependência linear permanece!}

\end{frame}




\begin{frame}[label=lild]{}

\begin{teo}
Sejam {\color{blue}$A$} e {\color{red}$B$} matrizes linha-equivalentes. 
Um conjunto qualquer de vetores coluna de {\color{blue}$A$} forma uma base do espaço coluna de {\color{blue}$A$} se, e só se, o conjunto de vetores correspondentes de {\color{red}$B$} forma uma base do espaço coluna de {\color{red}$B$}.
\end{teo}

\begin{exe}
Determine uma base para o espaço coluna da matriz
\[
A=\begin{bmatrix}
1 & -3 & 4 & -2 & 5 & 4\\
2 & -6 & 9 & -1 & 8 & 2\\
2 & -6 & 9 & -1 & 9 & 7\\
-1 & 3& -4 & 2 & -5 & -4
\end{bmatrix}.
\]
\end{exe}

\end{frame}

\begin{frame}[label=lild]{Aplicações: obtendo bases de subespaços gerados}
\begin{exe} Considere o subespaço $V$ de $\R^5$ gerado pelos vetores
\[\vec{v}_1=(1,-2,0,0,3), \vec{v}_2=(2,-5,-3,-2,6),
\]
\[\vec{v}_3=(0,5,15,10,0), \vec{v}_4=(2,6,18,8,6).
\]
\begin{enumerate}
\item Determine uma base qualquer de $V$.

\item Determine um subconjunto de $G=\{\vec{v}_1,\vec{v}_2,\vec{v}_3,\vec{v}_4\}$ que forma uma base de $V$.
\item Escreva os vetores de $G$ que não estão na base como combinação linear dos vetores da base.
\end{enumerate}

\end{exe}

\end{frame}


\begin{frame}[label=lild]
\begin{casa} Encontre um subconjunto dos vetores 
\[\vec{v}_1=(1,-2,0,3), \vec{v}_2=(2,-5,-3,6),
\]
\[\vec{v}_3=(0,1,3,0), \vec{v}_4=(2,-1,4,-7), \vec{v}_5=(5,-8,1,2)
\]
que forma uma base para o espaço gerado por eles. Escreva aqueles que não estão na base como uma combinação linear da base. 

\end{casa}
\end{frame}





