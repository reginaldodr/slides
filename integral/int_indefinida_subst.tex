\subsection*{Integral indefinida}
\begin{frame}
\frametitle{ Integral indefinida}
\begin{small}

\uncover<1->{A \dt{integral indefinida} de $f$ em relação a $x$ é o conjunto de todas as primitivas de uma função $f$ e denotamos da seguinte forma
$$\int f(x)dx.$$
O símbolo $\int$ é dito \dt{sinal de integração}, $f$ é dita \dt{integrando} da integral e $x$ é a \dt{variável de integração.}}

\uncover<2->{\begin{exe} Encontre as integrais indefinidas
$$\int \frac{1}{x}dx,\ \ \int e^x\ dx, \ \ \int\frac{1}{1+x^2}dx, \ \ \int \sec^2 dx.$$
\\

\end{exe}}

\end{small}
\end{frame}


\begin{frame}[fragile=singleslide]{Usando Python}
%\begin{small}
\begin{block}{ }
\begin{pyverbatim}
import sympy as sp

x = sp.symbols('x') #variável
f=1/x #função
g=1/(1+x**2) #função
intf=sp.integrate(f,x) #integral indefinida
intg=sp.integrate(g,x) #integral indefinida
\end{pyverbatim}
\end{block}
%\end{small}
\begin{pycode}
import sympy as sp
x = sp.symbols('x') #variável
f=1/x #função
g=1/(1+x**2) #função
intf=sp.integrate(f,x) #integral indefinida
intg=sp.integrate(g,x) #integral indefinida
\end{pycode}

\[\dps \int \frac{1}{x}\, dx=\py{sp.latex(intf)}+C,\ \dps \int \frac{1}{1+x^2}\, dx=\py{sp.latex(intg)}+C.\]

\begin{alertblock}{Observação}
Neste curso, reservaremos a notação $\log(x)$ para o logarítmo na base $e$, isto é, 
$\log(x)=\log_e(x)$. Veja a justificativa em \href{https://reginaldodr.github.io/academic/posts/notacao-log/notacao-log.html}{\beamergotobutton{Link}}
\end{alertblock}

\end{frame}


%\begin{frame}{Aplicações}
%
%\begin{exampleblock}{Queda livre de corpos}
%Consideremos um corpo de massa \textcolor{blue}{$m$} que é abandonado, a partir do repouso de uma altura \textcolor{blue}{$h_0$}. Desprezando-se a resistência do ar, determine uma função \textcolor{blue}{$h=h(t)$} que descreve a altura do objeto em cada instante do tempo. 
%\end{exampleblock}
%
%
%
%\end{frame}
\section{Técnicas de integração}

\subsection*{Regra da Substituição}

\begin{frame}
\frametitle{A regra da Substituição }
%\begin{small}

\uncover<1->{ \begin{teo} Se $u=g(x)$ é uma função derivável cuja imagem é um intervalo $I$ e  $f$ é contínua em $I$, então
$$\int f(g(x))g'(x)dx=\int f(u)du$$
\end{teo}}



\uncover<2->{\begin{exe} Calcule as integrais
\begin{multicols}{2}
\begin{enumerate}
\item $\dps\int \cos(x^2)2x\ dx$
\item $\dps\int e^{x^3}x^2\ dx$
%\item $\dps\int \frac{x}{x^2+1}\ dx$
\\

%\item $\dps\int_0^{\pi/4} \tg x\ dx$
\end{enumerate}
\end{multicols}
\end{exe}}


%\end{small}
\end{frame}

\begin{frame}
	\begin{casa}
		Calcule as integrais
			\begin{enumerate}
			\item $\dps\int \cos^2(x)\, dx$
			\item $\dps\int \sec(x)\, dx$
			\item $\dps\int\frac{x}{x^2+1}\, dx$
		\end{enumerate}
	\end{casa}
	

\end{frame}

%\begin{frame}{Aplicações}
%
%\begin{exampleblock}{Modelo Populacional Malthusiano}
%Este tipo de modelo é razoável para descrever populações que tem {\color{cyan}recurso ilimitados para crescimento e ausência de predadores}. 
%\begin{itemize}
%\item {\color{blue}$y(t)$}: número de indivíduos de uma população no instante $t$.
%\item {\color{red}$y'(t)$}: taxa de crescimento de uma população no instante $t$.
%
%\item Supõe-se que a {\color{red} taxa de crescimento} de uma população é proporcional à {\color{blue} população presente} 
%\[{\color{red}y'(t)}=k{\color{blue}y(t)}\]
%\end{itemize}
%
%
%\medskip
%
%Supondo que a população no instante $t=0$ é $y_0$, determine a função $y=y(t)$. Em quanto tempo a população dobra de tamanho?
%\end{exampleblock}
%\end{frame}
%
%\begin{frame}{Decaimento Radioativo}
%	Átomos instáveis podem emitir massa ou radiação espontaneamente, em um processo chamado {\color{blue}decaimento radioativo}. Isso pode resultar na formação de um novo elemento. Exemplos disso são o {\color{red}carbono-14 radioativo} que decai em nitrogênio e o rádio que decai em chumbo
%	
%	\begin{block}{Meia-vida}
%		A {\color{blue}meia-vida} de um elemento radioativo é o tempo necessário para que metade dos núcleos radioativos presentes em uma amostra decaiam. Por exemplo, sabe-se que {\color{red}a meia-vida do carbono-14 é 5730 anos}, ou seja, que em 5730 anos metade do carbono-14 presente transformou-se em  nitrogênio.
%	\end{block}
%	
%	Experimentos têm mostrado que em um determinado momento, {\color{blue}a taxa na qual um elemento radioativo decai é proporcional à quantidade de elemento presente}.
%	
%\end{frame}
%
%\begin{frame}{Datação por Carbono-14}
%	Uma ferramenta importante em pesquisa arqueológica é a {\color{red}datação por carbono-14} desenvolvida pelo químico estadunidense Willard F. Libby, que recebeu o prêmio Nobel de química em 1960 por este trabalho. 
%\medskip 
%	
%Em um organismo vivo, a proporção de {\color{red}carbono-14}, permanece relativamente constante durante a vida do organismo. Quando ele morre a absorção de {\color{red}carbono-14} cessa e a partir de então o {\color{red}carbono-14} {\color{blue}decai a uma taxa proporcional a quantidade presente}. 	
%	
%	\begin{casa}
%		Em um pedaço de madeira é encontrado $90\%$ da quantidade original de carbono-14. Qual a idade deste pedaço de madeira?
%	\end{casa}
%\end{frame}
%%\begin{frame}{Datação por Carbono 14}
%%
%%
%%A proporção de carbono 14(radioativo) em relação ao carbono 12 presentes nos seres vivos é constante. Quando um organismo morre a absorção de carbono 14 cessa e a partir de então o carbono 14 vai se transformando em carbono 12 a uma taxa proporcional a quantidade presente. Podemos descrever o problema de encontrar a quantidade de carbono 14 em função do tempo, $y(t)$, pela seguinte equação.
%%\[\frac{dy}{dt}=-ky\]
%%
%%\begin{exe} Em um pedaço de madeira é encontrado $1/500$ da quantidade original de carbono 14. Sabe-se que a meia-vida do carbono 14 é 5600 anos, ou seja, que em 5600 anos metade do carbono 14 presente transformou-se em carbono 12. Qual a idade deste pedaço de madeira?
%%\end{exe}
%%\end{frame}


\begin{frame}
\begin{small}
\uncover<1->{\begin{corol} Se $g' $ é contínua em $[a,b]$ e $f$ é contínua em $g([a,b])$, então
$$\int_a^b f(g(x))g'(x)dx=\int_{g(a)}^{g(b)} f(u)du$$
\end{corol}}

\uncover<2->{\begin{exe} Calcule 
 $\dps\int_{-1}^1 3x^2\sqrt{x^3+1}\ dx$

\end{exe} }


\end{small}
\end{frame}



%\begin{frame}
%\begin{small}
%
%\uncover<1->{\begin{prop} Seja $f$ contínua em $[-a,a]$.
%\begin{enumerate}[a]
%\item Se $f$ é par, então $\dps \int_{-a}^a f(x)dx=2\int_{0}^a f(x)dx$
%\item Se $f$ é ímpar, então $\dps \int_{-a}^a f(x)dx=0$
%\end{enumerate}
%\end{prop}}
%
%\uncover<2->{\begin{exe} 
%
%\begin{multicols}{2}
%\begin{enumerate}
%
%\item $\dps\int_{-\pi/2}^{\pi/2} \cos x\ dx$
%
%\item $\dps\int_{-\pi}^\pi \sen x\ dx$
%\\
%
%\end{enumerate}
%\end{multicols}
%
%
%\end{exe} }
%
%\end{small}
%\end{frame}



%\begin{frame}
%\frametitle{ Exemplos Importantes}
%\begin{exe} \begin{enumerate}
%\item $\dps\int\frac{dx}{\sqrt{8x-x^2}}$
%\medskip
%
%\item $\dps \int\frac{3x^2-7x}{3x+2}dx$
%\medskip
%
%\end{enumerate}
%\end{exe} 

%\end{frame}

\begin{frame}
	\begin{casa}
		Calcule as integrais
\begin{enumerate}
	\item  $\dps\int_0^{\pi/4} \tg x\ dx$
	
	\item $\dps\int\frac{dx}{\sqrt{8x-x^2}}$
		
%	\item $\dps \int\frac{3x^2-7x}{3x+2}dx$
	
\end{enumerate}
	\end{casa}
	
	
%\begin{exer}
%Calcule as integrais
%\begin{enumerate}
%\item $\dps \int \sen^3 (x)\cos^2 (x)\ dx$
%\item $\dps \int \sen (4x)\cos (5x)\ dx$
%\end{enumerate}
%\end{exer}
\end{frame}

%\begin{frame}[fragile=singleslide]{Usando Python}
%%\end{small}
%\begin{block}{title}
%\begin{pyverbatim}
%import sympy as sp
%x = sp.symbols('x')
%f=1/sp.sqrt(8*x-x**2)
%g=1/sp.sqrt(16-(x-4)**2)
%intf=sp.integrate(f,x)
%intg=sp.integrate(g,x)
%\end{pyverbatim}
%\end{block}
%
%\begin{pycode}
%import sympy as sp
%f=1/sp.sqrt(8*x-x**2)
%g=1/sp.sqrt(16-(x-4)**2)
%intf=sp.integrate(f,x)
%intg=sp.integrate(g,x)
%\end{pycode}
%
%\[=\py{sp.latex(intf)}+C.\]
%\end{frame}



%
%\begin{frame}
%\frametitle{ }
%\begin{small}
%
%\uncover<1->{\begin{exer} \begin{enumerate}
%\item Calcule a integral $\int (2x+1)^3 dx$ por dois métodos: 
%\begin{enumerate}[a]
%\item Desenvolvendo  o binômio $(2x+1)^3$.
%
%\item Fazendo a substituição $u=2x+1$. 
%
%Explique a aparente diferença obtida nas respostas.
%\end{enumerate}
%\medskip
%
%\item  Calcule a integral $\int \sqrt{x-1}x^2 dx$ por dois métodos: 
%\begin{enumerate}[a]
%\item Fazendo a substituição $u=x-1$.
%
%\item Fazendo a substituição $u=\sqrt{x-1}$. 
%
%Explique a aparente diferença obtida nas respostas.
%\end{enumerate}
%
%\end{enumerate}
%\end{exer} }
%
%\end{small}
%\end{frame}


