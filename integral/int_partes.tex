

\subsection*{Integração  por partes}


\begin{frame}
\frametitle{ Integração por partes}
\begin{small}

\uncover<1->{ \begin{teo} Se $f$ e $g$ têm derivadas contínuas, então
\begin{equation}\label{form_partes}\int f(x)g'(x)dx=f(x)g(x)-\int f'(x)g(x)dx.\end{equation}
\end{teo}}

\uncover<2->{Tomando $u=f(x)$ e $v=g(x)$, então $du=f'(x)dx$ e $dv=g'(x)dx$, assim a forma diferencial da equação (\ref{form_partes}) se torna
$$\int u\ dv=uv -\int v\ du$$}


\end{small}
\end{frame}


\begin{frame}
\frametitle{ }
\begin{small}

\uncover<1->{\begin{exe}
		Calcule as integrais abaixo:
\begin{enumerate}[a]
\item $\dps\int x\cos x\ dx$
\item $\dps \int \log x\ dx$
\item $\dps \int x^2e^x\ dx$
\item $\dps \int e^x \cos x\ dx$

\end{enumerate}


\end{exe} }

\end{small}
\end{frame}

%\begin{frame}{Fórmulas de redução}
%Usando a integração por partes é possível deduzir algumas fórmulas de redução, como por exemplo:
%\[\int \sin^n(x)\, dx=-\frac{1}{n}\cos(x)\sin^{n-1}(x)+\frac{n-1}{n}\int \sin^{n-2}(x)\,dx.\]
%
%\begin{exe}
%Calcule 
%\[\int\sin^4(x)\,dx.\]
%\end{exe}
%
%\end{frame}


\begin{frame}
	\begin{casa}
Calcule as integrais
\begin{enumerate}
%\item $\dps \int \sin^2(x)\, dx$
%\item $\dps \int \sen^3 (x)\cos^2 (x)\ dx$
%\item $\dps \int \sen^2 (x)\cos^2 (x)\ dx$

\item $\dps \int \sec^3 x\ dx$
%\item $\dps \int \sen (4x)\cos (5x)\ dx$
\end{enumerate}

\end{casa}



\end{frame}


%\begin{frame}
%\begin{small}
%
%\begin{casa}
%\begin{enumerate}
%
%\item Calcule a integral $\dps\int \sen(\ln x)dx$.
%
%\item Encontre a fórmula de recorrência para $\dps \int \sen^n x\ dx$ e calcule $\dps\int \sen^{10} x\ dx$.
%
%
%\end{enumerate}
%
%
%
%\end{casa}
%
%
%\end{small}
%
%
%\end{frame}

