%\subsection{Integrais Trigonométricas}
%
%
%\begin{frame}
%\frametitle{Integrais de funções Trigonométricas}
%\begin{small}
%
%\uncover<1->{\begin{block}{Produtos de potências de senos e cossenos}
%$$\int\sen^m x\cos^n x\ dx,$$
%onde $m$ e $n$ são inteiros não negativos.
%\end{block} }
%
%\uncover<2->{\begin{exe}
%\begin{enumerate}
%\item $\dps \int \sen^3 x\cos^2 x\ dx$
%\item $\dps \int \sen^2 x\cos^2 x\ dx$
%\end{enumerate}\end{exe}}
%
%\end{small}
%\end{frame}
%
%
%\begin{frame}
%\frametitle{ }
%\begin{small}
%
%\uncover<1->{\begin{block}{Produtos de potências de tangentes e secantes}
%$$\int \tg^m x\sec^n x\ dx,$$
%onde $m$ e $n$ são inteiros não negativos.
%\end{block} }
%
%\uncover<2->{\begin{exe}
%\begin{multicols}{2}
%\begin{enumerate}
%\item $\dps \int \tg^6 x\sec^4 x\ dx$
%\item $\dps \int \tg^5 x\sec^7 x\ dx$
%\item $\dps \int \sec^3 x\ dx$
%\bigskip
%
%\item $\dps \int \tg^2x \sec x\ dx$
%\item $\dps \int \tg^5 x \ dx$
%\\
%
%\end{enumerate}
%\end{multicols}\end{exe}}
%
%\end{small}
%\end{frame}
%
%\begin{frame}
%\frametitle{ }
%\begin{small}
%
%\uncover<1->{\begin{block}{Produtos de senos e cossenos}
%$$\int \sen mx \cos nx\ dx,\ \int \sen mx \sen nx\ dx,\ \int \cos mx \cos nx\ dx$$
%onde $m$ e $n$ são inteiros não negativos.
%\end{block} }
%
%\uncover<1->{\begin{block}{Fórmulas trigonométricas }
%$$\sen a\sen b=\frac{1}{2}\left(\cos(a-b)-\cos(a+b)\right)$$
%$$\cos a\cos b=\frac{1}{2}\left(\cos(a-b)+\cos(a+b)\right)$$
%$$\sen a\cos b=\frac{1}{2}\left(\sen(a-b)-\sen(a+b)\right)$$
%\end{block}  }
%
%\uncover<2->{\begin{exe}
%\begin{enumerate}
%\item $\dps \int \sen 4x\cos 5x\ dx$
%\end{enumerate}\end{exe}}
%
%\end{small}
%\end{frame}
%
%
%\begin{frame}
%\begin{small}
%\begin{exer}
%
%\begin{enumerate}
%
%\item Calcule a integral $\dps\int sen^2x\cos^4 x\ dx$.
%
%\item Encontre a fórmula de recorrência de $\dps\int \sec^n x\ dx$.
%
%\item Calcule a integral $\dps\int sen2x\sen4 x\ dx$.
%\end{enumerate}
%\end{exer}
%
%
%\end{small}
%\end{frame}
%
%
%\begin{frame}[label=int_trig,label=current]
%\frametitle{Substituição Trigonométrica }
%
%%\begin{block}{Substituição Trigonométrica}
%\begin{center}
%\begin{tikzpicture}[node distance = 1cm, thick]%
%\node (1) {{\color{blue}$\sqrt{1-x^2}$}};
%\node (2) [below=of 1] {$\sin^2(\theta)+\cos^2(\theta)=1$};
%\node[align=center] (2a) [below=of 2] {{\color{blue}$1-\sin^2(\theta)=\cos^2(\theta)$}\\ {\color{blue}$-\frac{\pi}{2}\leq \theta\leq \frac{\pi}{2}$}};
%
%\draw[->] (1) -- (2);
%\draw[->] (2) -- (2a);
%
%
%\node (3)[right=2cm of 1] {$\sqrt{1+x^2} \text{ ou } \sqrt{x^2-1}$};
%\node (4) [below=of 3] {$\sec^2(\theta)-\tan^2(\theta)=1 $};
%\draw[->] (3) -- (4);
%\end{tikzpicture}
%\end{center}
%%
%%\[\underbrace{\sqrt{1-x^2}}_{\sin^2(\theta)+\cos^2(\theta)=1} \ \ \ \underbrace{\sqrt{1+x^2} \text{ ou } \sqrt{x^2-1}}_{\sec^2(\theta)-\tan^2(\theta)=1}\]
%%\end{block}
%
%\end{frame}

\begin{frame}[label=int_trig]{Substituição Trigonométrica }

\begin{center}
\begin{tabular}{|c|l|l|}
\hline
 Expressão & Identidade & Substituição  \\ [1ex]
\hline 
& & \\
{\color{blue}$\sqrt{1-x^2}$} & {\color{blue}$1-\sin^2(\theta)=\cos^2(\theta)$}  & {\color{blue}$x=\sin(\theta),\ -\frac{\pi}{2}\leq \theta\leq \frac{\pi}{2}$}\\[1ex]
\hline
& & \\
{\color{red} $\sqrt{1+x^2}$} &  {\color{red} $1+\tan^2(\theta)=\sec^2(\theta)$}  & {\color{red}$x=\tan(\theta),\ -\frac{\pi}{2}< \theta< \frac{\pi}{2}$}  \\[1ex]
\hline
& & \\
{\color{red}$\sqrt{x^2-1}$} &   {\color{red} $\sec^2(\theta)-1=\tan^2(\theta)$}  & {\color{red}$x=\sec(\theta),$\ \makecell{$0\leq \theta< \frac{\pi}{2}$\\ ou \\ $\pi\leq \theta< \frac{3\pi}{2}$}}  \\
& & \\
\hline

\end{tabular}
\end{center}
\end{frame}

\begin{frame}[label=int_trig]

\begin{exe}
%\begin{multicols}{3}
\begin{enumerate}
\item Calcule $\dps\int\sqrt{1-x^2}\, dx$
\item Mostre que a área do círculo de raio \textcolor{blue}{$r$} é $\pi \textcolor{blue}{r}^2$.
%\item $\dps\int\frac{x^2dx}{\sqrt{9+x^2}}$
%\item $\dps\int\frac{dx}{\sqrt{25x^2-4}}$
%\medskip
\end{enumerate}
%\end{multicols}
\end{exe}

\end{frame}

\begin{frame}
\begin{casa}
Calcule
\[ \int\frac{dx}{x\sqrt{9+x^2}}\]
\end{casa}

\begin{exer}
Obtenha a fórmula da área da elipse $\frac{x^2}{a^2}+\frac{y^2}{b^2}=1$, onde $a,b>0$.
%\end{enumerate}
\end{exer}
\end{frame}


%\begin{frame}
%\begin{small}
%\begin{casa}
%Calcule as integrais
%\begin{enumerate}
%\item  $\dps\int \sqrt{1-9t^2} dt$.
%
%\item  $\dps\int_1^e \frac{dy}{y\sqrt{1+\ln^2y}}$.
%
%\item Obtenha a fórmula da área da elipse $\frac{x^2}{a^2}+\frac{y^2}{b^2}=1$, onde $a,b>0$.
%\end{enumerate}
%
%\end{casa}
%
%\end{small}
%
%
%\end{frame}




